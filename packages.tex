%%
%% packages.tex - User defined packages for edengths.tex
%%
%% Copyright (C) 2010 Mathew Topper <damm_horse@yahoo.co.uk>
%%
%% This file is part of the University of Edinburgh, Department of
%% Engineering LaTeX2e thesis template.
%% 
%% The University of Edinburgh, Department of Engineering LaTeX2e thesis
%% template is free software: you can redistribute it and/or modify
%% it under the terms of the GNU General Public License as published by
%% the Free Software Foundation, either version 3 of the License, or
%% (at your option) any later version.
%% 
%% The University of Edinburgh, Department of Engineering LaTeX2e thesis
%% template is distributed in the hope that it will be useful,
%% but WITHOUT ANY WARRANTY; without even the implied warranty of
%% MERCHANTABILITY or FITNESS FOR A PARTICULAR PURPOSE.  See the
%% GNU General Public License for more details.
%% 
%% You should have received a copy of the GNU General Public License
%% along with the University of Edinburgh, Department of Engineering
%% LaTeX2e thesis template.  If not, see <http://www.gnu.org/licenses/>.
%%
%%
%%   ABOUT
%%
%% This file contains the user defined packages for a Latex2e template which
%% corresponds to the regulations regarding layout of a thesis submitted within
%% the University of Edinburgh school of Engineering. It is not `official',
%% but conforms as best as possible to the regulation as detailed at:
%%
%%   http://www.scieng.ed.ac.uk/Postgraduate/PhD/settingoutyourthesis.htm
%%
%% Please feel free to alter the template to your own liking, but note that
%% the template is made available under the GNU GPL and must be similarly
%% licensed should you wish to release your modified template.
%%
%%
%%   CREDITS
%%
%% This template is an amalgamation of an existing Edinburgh University,
%% Electrical Engineering PhD Thesis class file (jthesis-v1.cls) authored by
%% George S Taylor which was released under the GNU GPL.
%% Code is included from the dmathesis class Written by M. Imran
%% for a thesis according to the university of Durham regulation, which was
%% released without copyright. It also contains ideas (possibly code) from the
%% Princeton thesis class file (PrincetonThesis.cls), authored by Mike Nolta.
%% Mathew Topper, Eoghan Maguire and Bill Edwards foresaw the need to maintain a
%% more recent latex implementation of the thesis regulations and thus, this
%% project was born. It is hoped that the template will be maintained by the
%% Edinburgh Engineering PhD community once released.

%%%% PACKAGES AUTOMATICALLY PROVIDED BY CLASS (No need to reload)

%% ifthen     -   Provides simple boolean commands

%% ifpdf      -   Detects whether pdflatex is being used.

%% graphicx   -   Allows inclusion of graphics in eps or jpg/pdf format

%% geometry   -   A more modern way of setting the page margins.
%%                'report' class options are passed automatically.

%% setspace   -   Define line spacing

%% appendix   -   Required for appendices

%% natbib     -   Natbib style bibliography with default options

%% ae         -   Nicer pdf output using T1 fonts

%% fontenc    -   T1 encoding stops some errors for unknown fonts

%% titlesec   -   Custom chapter titles and page headers

%% caption    -   Custom formatting of captions

%% subfig     -   Use subfigures in captions and formatting of such.

%% tocloft    -   Allow modifications to the table of contents and lists

%% mathptmx   -   Option 'msfonts' to use these alternative font packages.
%% helvet         Note: you could add your own font packages in this file.

%% showlabels -   Print labels on the page if 'labels' option is given.

%%%%%%%%%%%%%%%%%%%%%%%%%%%%%%%%%%%%%%%%%%%%%%%%%%%%%%%%%%%%%%%%%%%%%%%%%%
%%%%%          Put ADDITIONAL packages you want to use here           %%%%
%%%%%%%%%%%%%%%%%%%%%%%%%%%%%%%%%%%%%%%%%%%%%%%%%%%%%%%%%%%%%%%%%%%%%%%%%%

%% Allow coloured text. Particularly useful if using 'hyperref'.
%% Check out http://en.wikibooks.org/wiki/LaTeX/Colors for colours.
\usepackage[usenames]{color}

%% Typset URLs properly. This package does automatic breaking of long URLs.
%% If you're using hyperref you dont need this but you might need the
%% 'breakurl' package depending on your compilation route.
%% Note: 'breakurl' must be the last package (which is normally 'hyperref').
\usepackage{url} 

%% AMS Math Packages
\usepackage{amsmath}
\usepackage{amssymb}
\usepackage{amsthm}

%% Fancy chemical symbols (Might require installation)
% \usepackage[version=3]{mhchem}

%% Nicer tables
\usepackage{booktabs}

%% Proper hythenation
\usepackage[british,english]{babel}



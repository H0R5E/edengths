%\begin{centerchap}
\chapter*{Introduction}
\chaptermark{Introduction}
%\end{centerchap}

This is the beginning. I bet you wish that you had very started before long. I
shall now write two paragraphs.

This is the second paragraph.

In recent times the way in which the human race uses and generates its energy
has become of extreme
social, economical and environmental importance. Predictions of damaging
increases in mean global
temperatures \citep[see][]{Solomon:2007:CUP} has increased the need for carbon
emitting energy
technologies to be replaced by low carbon alternatives. Additional long term
economic factors are
also playing their part in shifting momentum to new technologies, in particular
the concerns about
oil and natural gas supplies. Demand for oil and gas is expected to outstrip
supply within the
current century, leading to inflated prices and energy security issues.
Unfortunately, although the
resource and environmental issues are occurring simultaneously, they are not
necessarily mutually
supportive. For instance, once the price for oil has reached a certain value it
can be economically
synthesised from coal, providing no environmental benefits. In fact, if the
market were left to
choose a method for replacing dwindling oil resource then this is one of the
most likely
substitutes; even in light of population growth, coal reserves are estimated to
last for hundreds of
years \citep[see][]{Jaccard:2005:CUP} and thus the cost is low.

With markets failing to deliver the necessary changes, it has become the
responsibility of
governments to intervene under the premise that the predicted environment and
economical
consequences of inaction will outweigh the costs incurred by immediate action.
This is a tough
political task as, particularly in the UK, energy markets have followed a trend
of liberalised
trading which makes strategic decision making extremely challenging. Private
investment is unlikely
to match policy, unless the policy is seen to be long term and economically
advantageous; a virtual
impossibility when scientific and economic opinions of the market requirements
are so uncertain. In
addition, the likelihood of direct governmental intervention in the form of
capital investment is
now looking more unlikely, due to the recent financial crisis. Capital spending
is set to fall
significantly in order to reduce deficits endured to rescue the banking sector.

Despite the potential funding problems, the United Kingdom (UK) Government's
Climate Change Act 2008
\citep*[see][2008]{CCA:2008:Defra} introduced for the first time legally binding
targets for
greenhouse gas emissions from within the UK. The targets set an 80\% reduction
of greenhouses gases
by 2050 and a 26\% reduction of carbon-dioxide (\ce{CO2}) emissions by 2020 with
respect to 1990
levels. This is set in the context of Britain's pre-existing target to reduce
\ce{CO2} emissions by
12\% as a ratified signatory of the Kyoto Protocol and a European Union proposal
to cut EU wide
greenhouse gas emissions by 20\% by 2020 each with respect to 1990 levels.
Electricity generation in
the United Kingdom accounts for 37\% of all \ce{CO2} emissions and as part of
the government's
greenhouse gas reduction strategy it will seek to reduce this to zero by 2030
\citep[see][2008]{CCC:2008}. Given the enormity of this task it is foreseen that
a varied mix of
low/zero carbon electricity generating technologies will be required.
\begin{quote}
 ``It is a well known fact that the use of a high-order panel method is more
accurate than the
low-order panel method or the discrete vortex method in computing the velocity
field as the
appearance of instabilities in the vortex sheet due to the spurious numerical
effects introduced by
a too-crude representation.''
\end{quote}
As such, a melting pot of renewable and sustainable energy technologies have
begun to compete to
become part of a `post-carbon' energy mix. Some of these technologies are well
established, such a
nuclear, biomass and wind, others less so, such as solar photovoltaic and marine
energy. In
addition, carbon emitting technologies such as coal remain an attractive option
for governments and
investors alike, as carbon capture and storage technology promises to store away
the greenhouse
gases emitted in combustion. However, carbon capture and storage is still an
unproven technology and
thus, the future costs of this and many other of the new energy technologies are
hard to predict and
are, at present, highly contestable; ultimately, over the long term, energy cost
is the most likely
deciding factor as to which technologies will feature most. Hence, each industry
is working hard to
reduce costs via research and innovation.

\chapter{pants}

Pant pant pants.

%\Blindtext

%\begin{centerchap}
\chapter{Superbob}
%\end{centerchap}

bob bob bob

\chapter{pant again}

More panty pants pants.
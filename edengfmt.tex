%%
%% edengfmt.tex - User defined formatting for thesis class
%%
%% Copyright (C) 2010 Mathew Topper <damm_horse@yahoo.co.uk>
%%
%% This file is part of the University of Edinburgh, Department of
%% Engineering LaTeX2e thesis template.
%% 
%% The University of Edinburgh, Department of Engineering LaTeX2e thesis
%% template is free software: you can redistribute it and/or modify
%% it under the terms of the GNU General Public License as published by
%% the Free Software Foundation, either version 3 of the License, or
%% (at your option) any later version.
%% 
%% The University of Edinburgh, Department of Engineering LaTeX2e thesis
%% template is distributed in the hope that it will be useful,
%% but WITHOUT ANY WARRANTY; without even the implied warranty of
%% MERCHANTABILITY or FITNESS FOR A PARTICULAR PURPOSE.  See the
%% GNU General Public License for more details.
%% 
%% You should have received a copy of the GNU General Public License
%% along with the University of Edinburgh, Department of Engineering
%% LaTeX2e thesis template.  If not, see <http://www.gnu.org/licenses/>.
%%
%%
%%   ABOUT
%%
%% This is the user defined formatting file for a Latex2e template which
%% corresponds to the regulations regarding layout of a thesis submitted within
%% the University of Edinburgh school of Engineering. It is not `official', but
%% conforms as best as possible to the regulation as detailed at:
%%
%%   http://www.scieng.ed.ac.uk/Postgraduate/PhD/settingoutyourthesis.htm
%%
%% Please feel free to alter the template to your own liking, but note that
%% the template is made available under the GNU GPL and must be similarly
%% licenced should you wish to release your modified template.
%%
%%   NOTE : THIS FILE MUST BE IN THE SAME DIRECTORY AS edengths.cls
%%
%%
%%   CREDITS
%%
%% This template is an amalgamtion of an existing Edinburgh University,
%% Electrical Engineering PhD Thesis class file (jthesis-v1.cls) authored by
%% George S Taylor which was released under the GNU GPL.
%% Code is included from the dmathesis class Written by M. Imran
%% for a thesis according to the university of Durham regulation, which was
%% released without copyright. It also contains ideas (possibly code) from the
%% Princeton thesis class file (PrincetonThesis.cls), authored by Mike Nolta.
%% Mathew Topper, Eoghan Maguire and Bill Edwards forsaw the need to maintain a
%% more recent latex implementation of the thesis regulations and thus, this
%% project was born. It is hoped that the template will be maintained by the
%% Edinburgh Engineering PhD community once released.
%%
%%
%%   RECORD OF REVISIONS
%%
%%     Date      Programmer        Description of change
%%     ====      ==========        =====================
%%   18/06/10    Mathew Topper     Made copy of format.tex and renamed.
%%
%%   19/06/10			   Added defaultfont to control the header
%%                                 font using class options. Also reduce
%%				   the size of the header separation.
%%
%%   21/06/10                      Controlling headers using titlesec rather
%%                                 than fancyhdr to avoid confusion.
%%
%%				   Added \raggedbottom to stop the bad boxes
%%				   when the twoside option is set.
%%
%%   24/06/10			   Added another pagestyle 'twomain' which can
%%				   be used to set different headers (for even
%%				   pages) if twoside option is set.
%%
%%   25/06/10                      Setting bibsep as rapport3 sets it to zero.
%%
%%                                 Added singlespace option.
%%
%%   08/07/10			   Put some plus minus into bibsep to reduce
%%				   the number of entries that span pages.
%%

%%%% SET THE MARGINS

%% Note, the margins here are University of Edinburgh, School of Engineering
%% regulation. You may wish to borrow a millimetre here or there and change
%% the header, for example, and thus the setting of the margins has been made
%% available here. It's not recommended you change these too much!
%% Note: Margins are controlled by the 'geometry' package.
\geometry{top=20mm, inner=40mm, outer=25mm, bottom=40mm}

%% Control the header height and separation for fancyhdr
\geometry{headheight=15.2pt, headsep=4mm}

%% Set the space between the main text and the start of the footnone
\geometry{footnotesep=5mm}


%%%% SET THE LINE SPACING.

%% Controlled by `setspace' package.
%% Note: you can defined environments for different spacings during the 
%% document. Quotes are automatically single spaced via this method.
%% ie \begin{singlespace}blah blah\end{singlespace}

%% Pick one of these three options for global spacing.

% \singlespacing                       % Single spaced

\onehalfspacing                        % One and a half spaced

% \doublespacing                       % Double spacing

% \setstretch{1.45}                    % Custom spacing. 1.3 ~ onehalf &
                                       % 1.6 ~ double spaced


%%%% VERTICAL SPACE IN ALIGNED EQUATIONS

%% Set a global spacing between vertically aligned equaions environments.
%% (To add local space after the break do //[some space ie 12pt])
% \setlength{\jot}{14pt}


%%%% TABLE OF CONTENTS AND LISTS

%% These commands are part of the `tocloft' package.

%% Remove the indent on the list of figures and tables.
\setlength{\cftfigindent}{0pt}
\setlength{\cfttabindent}{0pt}


%% Custom skip before chapters in the TOC. This is useful is, say, the
%% appendicies entry is alone at the foot of the page and you need to 
%% add a little more length.

% \addtolength{\cftbeforechapskip}{2pt}


%% If you don't like capitalised titles in the forematter you can changed these
%% here.

% \renewcommand{\listfigurename}{List of figures}
% \renewcommand{\listtablename}{List of tables}

%%%% BIBLIOGRAPHY

%% Set spacing between entrys (this varies between report and rapport3)
%% plus, minus reduces the number of entries which span pages.
\setlength{\bibsep}{2ex plus 1ex minus 1ex}

%% Rename bibliography should you desire
% \renewcommand{\bibname}{References}

%%%% FANCY HEADERS (using titlesec)

%% You can mess around with these if you like, but I wouldn't bother, really.
%% DON'T CHANGE THE NAMES 'main' AND 'twomain'.

%% Headers for the main body of the text, ie from contents onwards.
%% These headers are used when the document is single sided.
%% If there is no chapter number then uppercase lettering is used.
%% Format is {left}{center}{right}. \defaultfont sets the font to sans.
\newpagestyle{main}[\defaultfont\bfseries]{%
  \headrule%
  \sethead%
   {\ifthesection{\thesection{.} \space}{\ifthechapter{\thechapter{.}%
    \space}{}}\ifthesection{\sectiontitle}{\ifthechapter{\chaptertitle}{%
    \MakeUppercase{\chaptertitle}}}}%
   {}{\thepage}%
  \setfoot{}{}{}%
  }

%% If the twoside option is set, the even pages can have different styles.
%% These are defined in the sqaure brackets [left][centre][right]{}{}{}
%% The headers are now controlled by twomain pagstyle.
\newpagestyle{twomain}[\defaultfont\bfseries]{%
  \headrule%
  \sethead%
    [\thepage]%
    [][\ifthechapter{\chaptertitle}{\MakeUppercase{\chaptertitle}}]%
    {\ifthesection{\thesection{.} \space}{\ifthechapter{\thechapter{.}%
    \space}{}}\ifthesection{\sectiontitle}{\ifthechapter{\chaptertitle}{%
    \MakeUppercase{\chaptertitle}}}}%
    {}{\thepage}%
  \setfoot[][][]{}{}{}%
  }

%% Redefine the plain page style to use the defaultfont for the
%% numbering.
\renewpagestyle{plain}[\defaultfont]{%
  \sethead{}{}{}%
  \setfoot{}{\thepage}{}%
}

%% FIX VBOX ERROR WITH 'twoside' OPTION

%% Makes all pages the height of the text on that page
\raggedbottom


